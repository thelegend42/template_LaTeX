%ce code a été écrit par MICHEL POIRIER
%FICHIER CONTENANT TOUS LES PACKAGES UTILIES POUR LA RÉALISATION D'UN BEAU DOCUMENT LaTeX
%SIMPLEMENT FAIRE %ce code a été écrit par MICHEL POIRIER
%FICHIER CONTENANT TOUS LES PACKAGES UTILIES POUR LA RÉALISATION D'UN BEAU DOCUMENT LaTeX
%SIMPLEMENT FAIRE %ce code a été écrit par MICHEL POIRIER
%FICHIER CONTENANT TOUS LES PACKAGES UTILIES POUR LA RÉALISATION D'UN BEAU DOCUMENT LaTeX
%SIMPLEMENT FAIRE %ce code a été écrit par MICHEL POIRIER
%FICHIER CONTENANT TOUS LES PACKAGES UTILIES POUR LA RÉALISATION D'UN BEAU DOCUMENT LaTeX
%SIMPLEMENT FAIRE \input{STRUCTURE.tex} au début du "main.tex", juste après le document class

%packages concernant l'ENCODAGE et la FRANCISATION
\usepackage[utf8]{inputenc}
\usepackage{natbib} % le package de bibliographie doit être présent à cet endroit
\usepackage[french]{babel}
\usepackage[T1]{fontenc}

%MARGES du document
\usepackage[top=2cm,bottom=2cm,left=2cm,right=2cm]{geometry}

% mise en forme des PARAGRAPHES
\usepackage{setspace}
\usepackage[skip=0pt]{parskip}
\setlength{\parindent}{15pt}

\usepackage{cochineal} % ma POLICE préfére
\usepackage{lipsum} % générer du texte aléatoire
\usepackage{adjustbox} % ajuster les tailles/dimensions du texte et des tableaux

% pour formatter les titres des SECTIONS  
\usepackage{titlesec}
%\titlespacing*{\section}{0pt}{15pt}{0pt}
%\titlespacing*{\subsection}{0pt}{10pt}{0pt}
%\titlespacing*{\subsubsection}{0pt}{5pt}{0pt}

% notes de bas de page
\usepackage[bottom]{footmisc}

%packages de math et physique
\usepackage{physics}
\usepackage{amsmath}
\usepackage{bm}
\usepackage{mathrsfs}
\usepackage{amssymb}
\usepackage{amsthm}

% figures
\usepackage{graphicx}
\usepackage{subcaption}
\usepackage{wrapfig} % figures dans le texte
\usepackage{booktabs} % permet d'afficher des plus belles lignes pour les tableaux
\newcommand{\tabitem}{~~\llap{\textbullet}~~}
\usepackage{pgfplots}
\pgfplotsset{width=0.45\linewidth,compat=1.9}
\usepgfplotslibrary{fillbetween}
\usepackage{tikz}
\usetikzlibrary{arrows.meta,bbox}
\usepackage{diagbox}
\usepackage{lscape}

%PACKAGES VARIÉS
\usepackage{multicol} % environnement multicolonnes
\usepackage{appendix} % gestion des annexes
\usepackage{minted} % permet de mettre du code avec la syntaxe en couleur
\usepackage{enumitem} % pour énumérer (listes pimpées)
\usepackage{comment} % environnement de commentaires

% numérotation des équations
%\numberwithin{equation}{subsection} % numéroter les équations avec le # de sous-section

% noms des tableaux
\addto\captionsfrench{\def\tablename{Tableau}} % changer "Table" pour "Tableau"

% hyperliens en couleur
\usepackage[colorlinks=true, allcolors=black]{hyperref}
\newcommand{\changeurlcolor}[1]{\hypersetup{urlcolor=#1}}
\usepackage{url}


% shits de bibliographie
\setcitestyle{authoryear}
\usepackage{csquotes}

% faire un INDEX
\usepackage{makeidx}
\usepackage{tocbibind} % ajouter l'index et la biblio dans le ToC
\makeindex

%%%%%%%%%%%%%%%%%%%%%%%%%%%%COMMANDES%%%%%%%%%%%%%%%%%%%%%%%%%%%%%%%%%%
\newcommand{\HRule}{\rule{\linewidth}{0.3mm}}
\newenvironment{Figure}
  {\par\medskip\noindent\minipage{\linewidth}}
  {\endminipage\par\medskip}

\def \hfillx {\hspace*{-\textwidth} \hfill}

%\renewcommand{\d}{\text{d}} %commande pour les dérivées, mais pas vraiment besoin à cause du package physics
%\renewbibmacro{in:}{. Dans~:} % peut être utile dans certains formats de biblio
 au début du "main.tex", juste après le document class

%packages concernant l'ENCODAGE et la FRANCISATION
\usepackage[utf8]{inputenc}
\usepackage{natbib} % le package de bibliographie doit être présent à cet endroit
\usepackage[french]{babel}
\usepackage[T1]{fontenc}

%MARGES du document
\usepackage[top=2cm,bottom=2cm,left=2cm,right=2cm]{geometry}

% mise en forme des PARAGRAPHES
\usepackage{setspace}
\usepackage[skip=0pt]{parskip}
\setlength{\parindent}{15pt}

\usepackage{cochineal} % ma POLICE préfére
\usepackage{lipsum} % générer du texte aléatoire
\usepackage{adjustbox} % ajuster les tailles/dimensions du texte et des tableaux

% pour formatter les titres des SECTIONS  
\usepackage{titlesec}
%\titlespacing*{\section}{0pt}{15pt}{0pt}
%\titlespacing*{\subsection}{0pt}{10pt}{0pt}
%\titlespacing*{\subsubsection}{0pt}{5pt}{0pt}

% notes de bas de page
\usepackage[bottom]{footmisc}

%packages de math et physique
\usepackage{physics}
\usepackage{amsmath}
\usepackage{bm}
\usepackage{mathrsfs}
\usepackage{amssymb}
\usepackage{amsthm}

% figures
\usepackage{graphicx}
\usepackage{subcaption}
\usepackage{wrapfig} % figures dans le texte
\usepackage{booktabs} % permet d'afficher des plus belles lignes pour les tableaux
\newcommand{\tabitem}{~~\llap{\textbullet}~~}
\usepackage{pgfplots}
\pgfplotsset{width=0.45\linewidth,compat=1.9}
\usepgfplotslibrary{fillbetween}
\usepackage{tikz}
\usetikzlibrary{arrows.meta,bbox}
\usepackage{diagbox}
\usepackage{lscape}

%PACKAGES VARIÉS
\usepackage{multicol} % environnement multicolonnes
\usepackage{appendix} % gestion des annexes
\usepackage{minted} % permet de mettre du code avec la syntaxe en couleur
\usepackage{enumitem} % pour énumérer (listes pimpées)
\usepackage{comment} % environnement de commentaires

% numérotation des équations
%\numberwithin{equation}{subsection} % numéroter les équations avec le # de sous-section

% noms des tableaux
\addto\captionsfrench{\def\tablename{Tableau}} % changer "Table" pour "Tableau"

% hyperliens en couleur
\usepackage[colorlinks=true, allcolors=black]{hyperref}
\newcommand{\changeurlcolor}[1]{\hypersetup{urlcolor=#1}}
\usepackage{url}


% shits de bibliographie
\setcitestyle{authoryear}
\usepackage{csquotes}

% faire un INDEX
\usepackage{makeidx}
\usepackage{tocbibind} % ajouter l'index et la biblio dans le ToC
\makeindex

%%%%%%%%%%%%%%%%%%%%%%%%%%%%COMMANDES%%%%%%%%%%%%%%%%%%%%%%%%%%%%%%%%%%
\newcommand{\HRule}{\rule{\linewidth}{0.3mm}}
\newenvironment{Figure}
  {\par\medskip\noindent\minipage{\linewidth}}
  {\endminipage\par\medskip}

\def \hfillx {\hspace*{-\textwidth} \hfill}

%\renewcommand{\d}{\text{d}} %commande pour les dérivées, mais pas vraiment besoin à cause du package physics
%\renewbibmacro{in:}{. Dans~:} % peut être utile dans certains formats de biblio
 au début du "main.tex", juste après le document class

%packages concernant l'ENCODAGE et la FRANCISATION
\usepackage[utf8]{inputenc}
\usepackage{natbib} % le package de bibliographie doit être présent à cet endroit
\usepackage[french]{babel}
\usepackage[T1]{fontenc}

%MARGES du document
\usepackage[top=2cm,bottom=2cm,left=2cm,right=2cm]{geometry}

% mise en forme des PARAGRAPHES
\usepackage{setspace}
\usepackage[skip=0pt]{parskip}
\setlength{\parindent}{15pt}

\usepackage{cochineal} % ma POLICE préfére
\usepackage{lipsum} % générer du texte aléatoire
\usepackage{adjustbox} % ajuster les tailles/dimensions du texte et des tableaux

% pour formatter les titres des SECTIONS  
\usepackage{titlesec}
%\titlespacing*{\section}{0pt}{15pt}{0pt}
%\titlespacing*{\subsection}{0pt}{10pt}{0pt}
%\titlespacing*{\subsubsection}{0pt}{5pt}{0pt}

% notes de bas de page
\usepackage[bottom]{footmisc}

%packages de math et physique
\usepackage{physics}
\usepackage{amsmath}
\usepackage{bm}
\usepackage{mathrsfs}
\usepackage{amssymb}
\usepackage{amsthm}

% figures
\usepackage{graphicx}
\usepackage{subcaption}
\usepackage{wrapfig} % figures dans le texte
\usepackage{booktabs} % permet d'afficher des plus belles lignes pour les tableaux
\newcommand{\tabitem}{~~\llap{\textbullet}~~}
\usepackage{pgfplots}
\pgfplotsset{width=0.45\linewidth,compat=1.9}
\usepgfplotslibrary{fillbetween}
\usepackage{tikz}
\usetikzlibrary{arrows.meta,bbox}
\usepackage{diagbox}
\usepackage{lscape}

%PACKAGES VARIÉS
\usepackage{multicol} % environnement multicolonnes
\usepackage{appendix} % gestion des annexes
\usepackage{minted} % permet de mettre du code avec la syntaxe en couleur
\usepackage{enumitem} % pour énumérer (listes pimpées)
\usepackage{comment} % environnement de commentaires

% numérotation des équations
%\numberwithin{equation}{subsection} % numéroter les équations avec le # de sous-section

% noms des tableaux
\addto\captionsfrench{\def\tablename{Tableau}} % changer "Table" pour "Tableau"

% hyperliens en couleur
\usepackage[colorlinks=true, allcolors=black]{hyperref}
\newcommand{\changeurlcolor}[1]{\hypersetup{urlcolor=#1}}
\usepackage{url}


% shits de bibliographie
\setcitestyle{authoryear}
\usepackage{csquotes}

% faire un INDEX
\usepackage{makeidx}
\usepackage{tocbibind} % ajouter l'index et la biblio dans le ToC
\makeindex

%%%%%%%%%%%%%%%%%%%%%%%%%%%%COMMANDES%%%%%%%%%%%%%%%%%%%%%%%%%%%%%%%%%%
\newcommand{\HRule}{\rule{\linewidth}{0.3mm}}
\newenvironment{Figure}
  {\par\medskip\noindent\minipage{\linewidth}}
  {\endminipage\par\medskip}

\def \hfillx {\hspace*{-\textwidth} \hfill}

%\renewcommand{\d}{\text{d}} %commande pour les dérivées, mais pas vraiment besoin à cause du package physics
%\renewbibmacro{in:}{. Dans~:} % peut être utile dans certains formats de biblio
 au début du "main.tex", juste après le document class

%packages concernant l'ENCODAGE et la FRANCISATION
\usepackage[utf8]{inputenc}
\usepackage{natbib} % le package de bibliographie doit être présent à cet endroit
\usepackage[french]{babel}
\usepackage[T1]{fontenc}

%MARGES du document
\usepackage[top=2cm,bottom=2cm,left=2cm,right=2cm]{geometry}

% mise en forme des PARAGRAPHES
\usepackage{setspace}
\usepackage[skip=0pt]{parskip}
\setlength{\parindent}{15pt}

\usepackage{cochineal} % ma POLICE préfére
\usepackage{lipsum} % générer du texte aléatoire
\usepackage{adjustbox} % ajuster les tailles/dimensions du texte et des tableaux

% pour formatter les titres des SECTIONS  
\usepackage{titlesec}
%\titlespacing*{\section}{0pt}{15pt}{0pt}
%\titlespacing*{\subsection}{0pt}{10pt}{0pt}
%\titlespacing*{\subsubsection}{0pt}{5pt}{0pt}

% notes de bas de page
\usepackage[bottom]{footmisc}

%packages de math et physique
\usepackage{physics}
\usepackage{amsmath}
\usepackage{bm}
\usepackage{mathrsfs}
\usepackage{amssymb}
\usepackage{amsthm}

% figures
\usepackage{graphicx}
\usepackage{subcaption}
\usepackage{wrapfig} % figures dans le texte
\usepackage{booktabs} % permet d'afficher des plus belles lignes pour les tableaux
\newcommand{\tabitem}{~~\llap{\textbullet}~~}
\usepackage{pgfplots}
\pgfplotsset{width=0.45\linewidth,compat=1.9}
\usepgfplotslibrary{fillbetween}
\usepackage{tikz}
\usetikzlibrary{arrows.meta,bbox}
\usepackage{diagbox}
\usepackage{lscape}

%PACKAGES VARIÉS
\usepackage{multicol} % environnement multicolonnes
\usepackage{appendix} % gestion des annexes
\usepackage{minted} % permet de mettre du code avec la syntaxe en couleur
\usepackage{enumitem} % pour énumérer (listes pimpées)
\usepackage{comment} % environnement de commentaires

% numérotation des équations
%\numberwithin{equation}{subsection} % numéroter les équations avec le # de sous-section

% noms des tableaux
\addto\captionsfrench{\def\tablename{Tableau}} % changer "Table" pour "Tableau"

% hyperliens en couleur
\usepackage[colorlinks=true, allcolors=black]{hyperref}
\newcommand{\changeurlcolor}[1]{\hypersetup{urlcolor=#1}}
\usepackage{url}


% shits de bibliographie
\setcitestyle{authoryear}
\usepackage{csquotes}

% faire un INDEX
\usepackage{makeidx}
\usepackage{tocbibind} % ajouter l'index et la biblio dans le ToC
\makeindex

%%%%%%%%%%%%%%%%%%%%%%%%%%%%COMMANDES%%%%%%%%%%%%%%%%%%%%%%%%%%%%%%%%%%
\newcommand{\HRule}{\rule{\linewidth}{0.3mm}}
\newenvironment{Figure}
  {\par\medskip\noindent\minipage{\linewidth}}
  {\endminipage\par\medskip}

\def \hfillx {\hspace*{-\textwidth} \hfill}

%\renewcommand{\d}{\text{d}} %commande pour les dérivées, mais pas vraiment besoin à cause du package physics
%\renewbibmacro{in:}{. Dans~:} % peut être utile dans certains formats de biblio
