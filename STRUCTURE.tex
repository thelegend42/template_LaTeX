%ce code a été écrit par MICHEL POIRIER
%FICHIER CONTENANT TOUS LES PACKAGES UTILIES POUR LA RÉALISATION D'UN BEAU DOCUMENT LaTeX
%SIMPLEMENT FAIRE %ce code a été écrit par MICHEL POIRIER
%FICHIER CONTENANT TOUS LES PACKAGES UTILIES POUR LA RÉALISATION D'UN BEAU DOCUMENT LaTeX
%SIMPLEMENT FAIRE %ce code a été écrit par MICHEL POIRIER
%FICHIER CONTENANT TOUS LES PACKAGES UTILIES POUR LA RÉALISATION D'UN BEAU DOCUMENT LaTeX
%SIMPLEMENT FAIRE %ce code a été écrit par MICHEL POIRIER
%FICHIER CONTENANT TOUS LES PACKAGES UTILIES POUR LA RÉALISATION D'UN BEAU DOCUMENT LaTeX
%SIMPLEMENT FAIRE \input{STRUCTURE.tex} au début du "main.tex", juste après le document class

%packages concernant l'ENCODAGE et la FRANCISATION
\usepackage[utf8]{inputenc}
\usepackage[french]{babel}
\usepackage[T1]{fontenc}

%MARGES
\usepackage[top=2.5cm,bottom=2.5cm,left=2.5cm,right=2.5cm]{geometry}

%packages de math et physique
\usepackage{physics}
\usepackage{amsmath}
\usepackage{amssymb}

%FIGURES
\usepackage{graphicx, subcaption}
\usepackage{wrapfig}
\usepackage{booktabs} %permet d'afficher des plus belles lignes pour les tableaux

%PACKAGES VARIÉS
\usepackage{multicol} %environnement multicolonnes
\usepackage{appendix} %gestion des annexes
\usepackage{minted} %permet de mettre du code avec la syntaxe en couleur

%hyperliens en couleur
\usepackage[colorlinks=true, linkcolor=red, urlcolor=green, citecolor=blue, anchorcolor=blue]{hyperref}
\newcommand{\changeurlcolor}[1]{\hypersetup{urlcolor=#1}}

%packages pour la BIBLIOGRAPHIE
\usepackage[backend=biber,style=numeric,sorting=none]{biblatex} %pour la numérotation en ordre d'apparition
\usepackage{csquotes}
\addbibresource{bibliographie.bib} %nom du fichier.bib

%%%%%%%%%%%%%%%%%%%%%%%%%%%%COMMANDES%%%%%%%%%%%%%%%%%%%%%%%%%%%%%%%%%%
\renewcommand{\d}{\text{d}} %commande pour les dérivées

\newcommand{\HRule}{\rule{\linewidth}{0.2mm}}
\newenvironment{Figure}
  {\par\medskip\noindent\minipage{\linewidth}}
  {\endminipage\par\medskip}

\def \hfillx {\hspace*{-\textwidth} \hfill}
 au début du "main.tex", juste après le document class

%packages concernant l'ENCODAGE et la FRANCISATION
\usepackage[utf8]{inputenc}
\usepackage[french]{babel}
\usepackage[T1]{fontenc}

%MARGES
\usepackage[top=2.5cm,bottom=2.5cm,left=2.5cm,right=2.5cm]{geometry}

%packages de math et physique
\usepackage{physics}
\usepackage{amsmath}
\usepackage{amssymb}

%FIGURES
\usepackage{graphicx, subcaption}
\usepackage{wrapfig}
\usepackage{booktabs} %permet d'afficher des plus belles lignes pour les tableaux

%PACKAGES VARIÉS
\usepackage{multicol} %environnement multicolonnes
\usepackage{appendix} %gestion des annexes
\usepackage{minted} %permet de mettre du code avec la syntaxe en couleur

%hyperliens en couleur
\usepackage[colorlinks=true, linkcolor=red, urlcolor=green, citecolor=blue, anchorcolor=blue]{hyperref}
\newcommand{\changeurlcolor}[1]{\hypersetup{urlcolor=#1}}

%packages pour la BIBLIOGRAPHIE
\usepackage[backend=biber,style=numeric,sorting=none]{biblatex} %pour la numérotation en ordre d'apparition
\usepackage{csquotes}
\addbibresource{bibliographie.bib} %nom du fichier.bib

%%%%%%%%%%%%%%%%%%%%%%%%%%%%COMMANDES%%%%%%%%%%%%%%%%%%%%%%%%%%%%%%%%%%
\renewcommand{\d}{\text{d}} %commande pour les dérivées

\newcommand{\HRule}{\rule{\linewidth}{0.2mm}}
\newenvironment{Figure}
  {\par\medskip\noindent\minipage{\linewidth}}
  {\endminipage\par\medskip}

\def \hfillx {\hspace*{-\textwidth} \hfill}
 au début du "main.tex", juste après le document class

%packages concernant l'ENCODAGE et la FRANCISATION
\usepackage[utf8]{inputenc}
\usepackage[french]{babel}
\usepackage[T1]{fontenc}

%MARGES
\usepackage[top=2.5cm,bottom=2.5cm,left=2.5cm,right=2.5cm]{geometry}

%packages de math et physique
\usepackage{physics}
\usepackage{amsmath}
\usepackage{amssymb}

%FIGURES
\usepackage{graphicx, subcaption}
\usepackage{wrapfig}
\usepackage{booktabs} %permet d'afficher des plus belles lignes pour les tableaux

%PACKAGES VARIÉS
\usepackage{multicol} %environnement multicolonnes
\usepackage{appendix} %gestion des annexes
\usepackage{minted} %permet de mettre du code avec la syntaxe en couleur

%hyperliens en couleur
\usepackage[colorlinks=true, linkcolor=red, urlcolor=green, citecolor=blue, anchorcolor=blue]{hyperref}
\newcommand{\changeurlcolor}[1]{\hypersetup{urlcolor=#1}}

%packages pour la BIBLIOGRAPHIE
\usepackage[backend=biber,style=numeric,sorting=none]{biblatex} %pour la numérotation en ordre d'apparition
\usepackage{csquotes}
\addbibresource{bibliographie.bib} %nom du fichier.bib

%%%%%%%%%%%%%%%%%%%%%%%%%%%%COMMANDES%%%%%%%%%%%%%%%%%%%%%%%%%%%%%%%%%%
\renewcommand{\d}{\text{d}} %commande pour les dérivées

\newcommand{\HRule}{\rule{\linewidth}{0.2mm}}
\newenvironment{Figure}
  {\par\medskip\noindent\minipage{\linewidth}}
  {\endminipage\par\medskip}

\def \hfillx {\hspace*{-\textwidth} \hfill}
 au début du "main.tex", juste après le document class

%packages concernant l'ENCODAGE et la FRANCISATION
\usepackage[utf8]{inputenc}
\usepackage[french]{babel}
\usepackage[T1]{fontenc}

%MARGES
\usepackage[top=2.5cm,bottom=2.5cm,left=2.5cm,right=2.5cm]{geometry}

%packages de math et physique
\usepackage{physics}
\usepackage{amsmath}
\usepackage{amssymb}

%FIGURES
\usepackage{graphicx, subcaption}
\usepackage{wrapfig}
\usepackage{booktabs} %permet d'afficher des plus belles lignes pour les tableaux

%PACKAGES VARIÉS
\usepackage{multicol} %environnement multicolonnes
\usepackage{appendix} %gestion des annexes
\usepackage{minted} %permet de mettre du code avec la syntaxe en couleur

%hyperliens en couleur
\usepackage[colorlinks=true, linkcolor=red, urlcolor=green, citecolor=blue, anchorcolor=blue]{hyperref}
\newcommand{\changeurlcolor}[1]{\hypersetup{urlcolor=#1}}

%packages pour la BIBLIOGRAPHIE
\usepackage[backend=biber,style=numeric,sorting=none]{biblatex} %pour la numérotation en ordre d'apparition
\usepackage{csquotes}
\addbibresource{bibliographie.bib} %nom du fichier.bib

%%%%%%%%%%%%%%%%%%%%%%%%%%%%COMMANDES%%%%%%%%%%%%%%%%%%%%%%%%%%%%%%%%%%
\renewcommand{\d}{\text{d}} %commande pour les dérivées

\newcommand{\HRule}{\rule{\linewidth}{0.2mm}}
\newenvironment{Figure}
  {\par\medskip\noindent\minipage{\linewidth}}
  {\endminipage\par\medskip}

\def \hfillx {\hspace*{-\textwidth} \hfill}
